\input source_header.tex

\begin{document}
	%%%%%%%%%%%%%%%%%%%%%%%%%%%%%%%%%%%%%%%%%%%%%%%
	\docheader{2023}{Source}{\S 2 WGSL}{Martin Henz, Wang Ziwen}
	%%%%%%%%%%%%%%%%%%%%%%%%%%%%%%%%%%%%%%%%%%%%%%%

\input source_intro.tex

\section{Changes}

Source \S 2 WGSL utilizes the WebGPU API to accelerate GPU-based operations for Source programs. 
This modern web standard enables efficient rendering and data processing directly from web browsers. 
Specifically, in Source \S 2 WGSL, when a Source program invokes the play function from the Sound module, 
the wave function of the sound to be played undergoes partial evaluation and is transpiled into the WebGPU Shading Language (WGSL). 
This process significantly enhances sound processing speed on the GPU using WebGPU APIs. 
Comparative experiments have demonstrated that Source \S 2 WGSL achieves significantly higher performance speeds, 
being orders of magnitude faster than Source \S 2 for sound processing tasks.

\input source_bnf.tex

\input source_2_bnf.tex

\newpage

\input source_return

\input source_import

\input source_boolean_operators

\input source_names_lang

\input source_numbers

\input source_strings

\input source_comments

\input source_typing

\section{Standard Libraries}

The following libraries are always available in this language.

\input source_misc

\input source_math

\input source_lists

\input source_sound_gpu

\input source_js_differences

\newpage

\input source_list_library

\end{document}